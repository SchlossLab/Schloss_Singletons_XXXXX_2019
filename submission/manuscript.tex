\documentclass[]{article}
\usepackage[T1]{fontenc}
\usepackage{lmodern}
\usepackage{amssymb,amsmath}
\usepackage{ifxetex,ifluatex}
\usepackage{fixltx2e} % provides \textsubscript
% use upquote if available, for straight quotes in verbatim environments
\IfFileExists{upquote.sty}{\usepackage{upquote}}{}
\ifnum 0\ifxetex 1\fi\ifluatex 1\fi=0 % if pdftex
  \usepackage[utf8]{inputenc}
\else % if luatex or xelatex
  \ifxetex
    \usepackage{mathspec}
    \usepackage{xltxtra,xunicode}
  \else
    \usepackage{fontspec}
  \fi
  \defaultfontfeatures{Mapping=tex-text,Scale=MatchLowercase}
  \newcommand{\euro}{€}
\fi
% use microtype if available
\IfFileExists{microtype.sty}{\usepackage{microtype}}{}
\usepackage[margin=1.0in]{geometry}
\usepackage{graphicx}
% Redefine \includegraphics so that, unless explicit options are
% given, the image width will not exceed the width of the page.
% Images get their normal width if they fit onto the page, but
% are scaled down if they would overflow the margins.
\makeatletter
\def\ScaleIfNeeded{%
  \ifdim\Gin@nat@width>\linewidth
    \linewidth
  \else
    \Gin@nat@width
  \fi
}
\makeatother
\let\Oldincludegraphics\includegraphics
{%
 \catcode`\@=11\relax%
 \gdef\includegraphics{\@ifnextchar[{\Oldincludegraphics}{\Oldincludegraphics[width=\ScaleIfNeeded]}}%
}%
\ifxetex
  \usepackage[setpagesize=false, % page size defined by xetex
              unicode=false, % unicode breaks when used with xetex
              xetex]{hyperref}
\else
  \usepackage[unicode=true]{hyperref}
\fi
\hypersetup{breaklinks=true,
            bookmarks=true,
            pdfauthor={},
            pdftitle={},
            colorlinks=true,
            citecolor=blue,
            urlcolor=blue,
            linkcolor=magenta,
            pdfborder={0 0 0}}
\urlstyle{same}  % don't use monospace font for urls
\setlength{\parindent}{0pt}
\setlength{\parskip}{6pt plus 2pt minus 1pt}
\setlength{\emergencystretch}{3em}  % prevent overfull lines
\setcounter{secnumdepth}{0}

\author{}
\date{\vspace{-2.5em}}
\usepackage{booktabs}
\usepackage{longtable}
\usepackage{array}
\usepackage{multirow}
\usepackage{wrapfig}
\usepackage{float}
\usepackage{colortbl}
\usepackage{pdflscape}
\usepackage{tabu}
\usepackage{threeparttable}
\usepackage{threeparttablex}
\usepackage[normalem]{ulem}
\usepackage{makecell}
\usepackage{setspace}
\doublespacing
\usepackage[left]{lineno}
\linenumbers
\modulolinenumbers
\usepackage{helvet} % Helvetica font
\renewcommand*\familydefault{\sfdefault} % Use the sans serif version of the font
\usepackage[T1]{fontenc}
\usepackage{booktabs}
\usepackage{longtable}
\usepackage{array}
\usepackage{multirow}
\usepackage{wrapfig}
\usepackage{float}
\usepackage{colortbl}
\usepackage{pdflscape}
\usepackage{tabu}
\usepackage{threeparttable}
\usepackage{threeparttablex}
\usepackage[normalem]{ulem}
\usepackage{makecell}

\begin{document}

\section{Removal of rare sequences from 16S rRNA gene sequence surveys
biases the interpretation of community structure
data}\label{removal-of-rare-sequences-from-16s-rrna-gene-sequence-surveys-biases-the-interpretation-of-community-structure-data}

\vspace{35mm}

Patrick D. Schloss${^\dagger}$

\vspace{40mm}

$\dagger$ To whom correspondence should be addressed:

\href{mailto:pschloss@umich.edu}{pschloss@umich.edu}

Department of Microbiology and Immunology University of Michigan Ann
Arbor, MI 48109

\vspace{35mm}

\subsection{Research article format}\label{research-article-format}

\newpage
\linenumbers

\subsection{Abstract}\label{abstract}

\textbf{word choice: sample vs.~community}

250 words

\newpage

\subsection{Importance}\label{importance}

150 words

\newpage

\textbf{EDIT: Two elements that are held in tension in the analysis of
16S rRNA gene sequence data is how to adequately remove PCR and
sequencing artifacts and decrease the granularity of the taxonomic level
that is used in the analysis.} Previous attempts have included screening
for sequencing quality based on quality scores {[}Kozich/Edgar{]}
followed by a polishing step based on the frequency of the sequences
relative to similar sequences {[}Kozich/Edgar/DeBlur{]}. Other pipelines
model the quality scores and types of errors to cluster sequences
directly {[}Dada2{]}. But, as a final step many pipelines advocate for
removing rare sequences from each dataset prior to outputting the
sequence data as amplicon sequence variants (ASVs)
{[}Knight/Edgar/DeBlur/Dada2{]}. ASVs are often clustered further to
generate operational taxonomic units or phylotypes. Some pipelines
remove all ASVs that appear once (i.e.~singletons) {[}XXXX{]}, XXXXXX
{[}XXXXXX{]}, XXXXX {[}XXXXXXX{]}, or XXXXXX {[}XXXXXX{]} times prior to
further clustering or making ecological comparisions. Notably, the
mothur-based pipeline discourages the practice of removing rare
sequences.

The abundance-based screening approach assumes that rare ASVs are more
likely to be artifacts than more abundant ASVs. Sequencing of mock
communities confirms that artifacts tend to be rare. Proponents of
abundance-based screening point to their ability to obtain the correct
number of ASVs, OTUs, or phylotypes with data generated from sequencing
mock communities when rare ASVs are removed. However, this approach
effectively overfits the curation pipeline to data generated from a
phylogenetically simple community with an atypical community
distribution that is often sequenced to a depth that is not achieved
with biological samples. It is necessary to think more deeply about the
practice of abundance-based screening.

The minimum abundance thresholds that have been proscribed are developed
and applied without regard for the total number of sequences generated
from each sample. Ignored in their recommendations is the common
experience that the number of sequences generated from each sample may
vary by two or three orders of magnitude. An ASV that appears once in a
sample with 2,000 sequences is more trustworthy than an ASV that appears
once in a sample with 100,000 sequences since it has a 50-fold higher
relative abundance. But, according to the pipeline recommendations, they
are treated as being equally trustworthy. Rather than removing rare
ASVs, the approach taken by the mothur pipeline applies the classical
ecological approach of rarefaction. Each sample is rarefied to the same
sequencing depth so that the number of artifacts that appears in each
sample is controlled.

Experience sequencing biological samples demonstrates that there are
good ASVs that may have an abundance below the proscribed threshold. For
example, the abundance of an ASV may be below the threshold in some
samples or time points and above the threshold in others. However,
rarity, both in terms of prevalence and incidence, is an important
ecological concept. Removing rare ASVs likely hinders one's ability to
ability to make inferences about the dynamics and nature of the
populations that rare ASVs represent. Furthermore, removing ASVs whose
abundances are below the proscribed threshold also potentially biases
the community structure of the samples.

In the current study, I use published sequence data from 12 studies to
investigate the nature of rare ASVs (i.e.~those that appear 10 or fewer
times) and the effects of removing them on downstream analysis of
microbial communities. The analysis was also performed using traditional
operational taxonomic units, where ASVs subjected to abundance-based
screening were clustered such that the ASVs within an OTU were no more
than 3\% different from each other. The results reject the assumptions
built into abundance-based screening and highlight the problems inherent
in removing rare ASVs.

\subsection{Results}\label{results}

\textbf{Datasets.} I collected 12 publicly available datasets that used
the Illumina MiSeq platform to sequence the V4 region of the 16S rRNA
gene from a variety of environments (Table 1). To insure the highest
possible data quality, datasets were limited to those where the 500
cycle v2 MiSeq chemistry was used to sequence the amplicons. The paired
250 nt reads resulted in near complete 2-fold sequencing coverage of
every nucleotide in the ca. 250 nt-long region. This region and
sequencing platform were selected because previous work has shown that a
standard data analysis pipeline in mothur results in a sequencing error
rate below 0.02\% (1). All sequence data were obtained from the Sequence
Read Archive and processed using a standard mothur-based sequencing
pipeline that resulted in ASVs as generated by the pre.cluster algorithm
(1, 2). After removing poor quality and chimeric ASVs and samples that
had uncharacteristically low number of sequences for the dataset, these
datasets included between 7 and 490 samples (Figure S1). The median
number of sequences for each dataset ranged between 6,477 and 193,464
(Table 1). Strikingly, aside from the relatively small marine and soil
datasets, the difference between the sample with the fewest sequences
and the sample with the most sequences for each dataset varied by
between 7.4 and 96.6-fold (Table 1).

\textbf{The nature of singletons.} Removal of rare ASVs is commonly
justified as a method of removing ASVs that are artifacts. If such ASVs
are artifacts, then one would expect the number of singleton ASVs to
accumulate with sequencing depth. Contrary to this expectation, the
median percentage of sequences that were discarded when singleton ASVs
were removed from each dataset varied between 0.42 and 22.23\%
(bioethanol and seagrass). In addition, with the exception of the
samples from the marine and sediment datasets (Spearman correlation,
P\textgreater{}0.05), the fraction of singleton ASVs in samples was
negatively correlated with the number of sequences in each sample with a
range between -0.27 and -0.87 (rice and bioethanol) (Figure 1A). This
showed that with additional sequencing, the probability of seeing
singleton ASVs in multiple samples was greater than the probability of
generating an artifact. This suggests that the singleton ASVs are not as
likely to be artifacts as previously thought. Furthermore, if singleton
ASVs were artifacts, then one would not expect to find them in other
samples from the same dataset. In fact, singleton ASVs from samples with
fewer sequences were often found in samples with more sequences. At
least 50\% of the singleton ASVs found in the samples from the mice,
rice, seagrass, and stream datasets were found in another sample from
the same dataset (Figure 1B). Considering the likelihood of finding an
ASV duplicated in another sample is confounded by the number of samples
and inter-sample diversity, the high coverage of singleton ASVs in these
datasets was remarkable. The correlation between the number of sequences
in a sample and the fraction of that sample's singleton ASVs that were
covered by another sample in the dataset was significant and negative
for 9 of the datasets ranging between -0.31 and -0.84 for the rice and
seagrass datasets, respectively (Figure 1C). The negative correlation
indicated that the singleton ASVs in the smaller samples were more
likely to be covered by ASVs in the larger samples. Among the three
datasets without a significant correlation (Spearman correlation,
P\textgreater{}0.05), the marine and soil datasets had the fewest
samples in our collection and the stream dataset already had a high
level of coverage regardless of the number of sequences. Contrary to the
common motivation for removing rare ASVs, these results indicate that
this practice disproportionately impacts samples with fewer sequences
and likely removes more non-artifact ASVs than those that are artifacts.

\textbf{The impact of removing rare ASVs on the information represented
in each sample.} Removing rare ASVs will reduce the richness of ASVs and
proportionally increase the relative abundance of the remaining ASVs.
The result was expected to be a loss of information contained within
each sample. To quantify the effect of removing rare ASVs on the
information contained within each sample, I varied the minimum abundance
threshold to simulate removing ASVs of varying rarity from each sample.
The richness of ASVs in each sample (i.e.~the number of ASVs) decreased
by between 34.4 and 86.2\% when removing those ASVs that only appeared
once and by between 76.0 and 95.6\% when removing those that appeared
ten or fewer times from each sample (Figure 2A). Similarly, the Shannon
diversity decreased by between 1.8 and 15.9\% when removing ASVs that
only appeared once and by between 76.0 and 95.6\% when removing ASVs
that appeared ten or fewer times from each sample (Figure 2B). Next, I
assigned the ASVs to OTUs to assess the impact of removing rare ASVs on
higher level taxonomic groupings that are commonly used in microbial
ecology studies. Although pooling similar ASVs into OTUs reduced the
impact of removing the rare ASVs relative to the ASV-based analysis, the
minimum abundance threshold still decreased the richness of OTUs and the
diversity decreased relative to the full community (\textbf{Figure
S2AB}). In contrast to the richness and diversity measurements, the
Kullback--Leibler divergence compares the relative abundance of specific
ASVs or OTUs between representations of the community. I calculated the
Kullback--Leibler divergence from the full to pruned communities when
rare ASVs were removed. As the threshold for removing ASVs increased,
the amount of information lost also increased for both ASVs and OTUs
(Figure 2C and Figure S2C). The relative loss of information was
generally lower for OTUs than than it was for ASVs. Removing rare ASVs,
regardless of abundance threshold, had profound impacts on the
representation of the communities.

\textbf{Removing treatment group effects from community data.} Because
treatment effects often affect a sample's diversity and inter-sample
variation, I generated null distributions for each study by randomizing
the number of times each ASV was observed in each sample such that the
total number of sequences in each sample and the total number of times
each ASV was observed across all samples in the study was the same as
was originally observed. This effectively made every community in a
study a statistical sample of the study-wide composite community
distribution. For example, after this procedure, the 490 samples from
the human dataset would be expected to have the same richness and
diversity of ASVs and one would not expect to find treatment-based
effects between the samples. Because of the risk of bias if only one
representation of the null distribution was generated, I generated 100
randomized datasets for each study. The trends between removing rare
ASVs and the richness, diversity, and information loss that were
identified using the observed community community distribution data were
also identified with the data from the null distribution; however, the
losses were larger when using the null distribution data (\textbf{Figure
S3}). The null distribution data were used in the remainder of the study
to minimize the risk of bias.

\textbf{The impact of removing rare ASVs on the information represented
between samples.} Considering the loss of richness, diversity, and
information when a community has its rarest ASVs removed, it seemed
likely that the relationship between communities would also be altered.
To assess the impact of removing rare ASVs on measures of alpha
diversity between samples I calculated the coefficients of variation
(COVs, i.e.~standard deviation divided by the mean) for richness and
diversity for each study at multiple abundance thresholds. The COVs for
richness of ASVs across the studies after removing singletons were
between 3.6 and 32.7-times larger than they were without removing
singleton ASVs (Figure 3A). Similarly, the COVs for the diversity of
ASVs were between 1.8 and 20.4-times larger when singletons were removed
than when they were not removed (Figure 3B). To assess the impact of
removing rare ASVs on measures of beta diversity between samples I
calculated the COVs of the Bray-Curtis distances between samples within
the same study at multiple abundance thresholds. The COVs between
Bray-Curtis distances within a study when singletons were removed was
between 1.3 and 18.6-times larger than when they were not removed
(Figure 3C). Increasing the minimum abundance threshold increased the
COVs between samples when using metrics of alpha and beta diversity.
When ASVs were clustered into OTUs the difference in COVs was less than
it was for the ASVs (Figure S4). These results indicate that removing
rare ASVs increases the dissimilarity between samples, which could have
a significant impact on the statistical power to detect differences
between treatment groups.

\textbf{The impact of removing rare ASVs on the ability to detect
statistically significant differences between treatment groups.} To test
the effect of increased inter-sample variation, I randomly assigned
samples to one of two treatment groups. In the first treatment group,
communities were randomly sampled from the null distribution as
described above. For the second treatment group, I increased the
abundance of 10\% of the ASVs in the pooled study distribution by 5\%. I
randomly generated 100 simulated sets of treatment groups and samples. I
then tested the ability to detect a difference between the two treatment
groups using alpha and beta diversity metrics. The fraction of
significant tests was a measurement of the statistical power to detect
the difference between the treatment groups. When considering the
differences in richness and diversity, the marine dataset yielded no
simulated sets that were statistically significant, which was likely due
to the small number of samples in the study (N=7). Among the remaining
datasets, the power to detect a difference in the richness of ASVs
ranged between 0.10 and 0.49 and between 0.10 and 0.53 to detect a
difference in diversity when using a Wilcox test (Figure 4A). When
singleton ASVs were removed, the power to detect a difference in the
diversity of ASVs dropped by between 27.3 and 92.9\% and by between 40.0
and 93.3\% (Figure 4B). The effect of removing rare ASVs on the richness
of OTUs and their diversity was similar (Figure S5AB). I used the
Bray-Curtis dissimilarity index to compare the simulated communities
within each dataset and calculated the power to detect differences
between the two simulated treatment groups using the analysis of
molecular variance (also called PERMANOVA) (Figure 4C and S5C). Without
removing rare sequences, the power to detect a difference between the
two simulated treatment groups varied between 0.41 and 1.00. Aside from
3 datasets, the power to detect differences dropped by between 6.5 and
64.0\% when singletons were removed. However, when ASVs that occurred 10
or fewer times were removed from each sample, the power to detect
differences dropped by 12.0 and 97.2\%; similar results were observed
when ASVs were clustered into OTUs. Removing rare ASVs reduced the
ability to detect simulated treatment effects using metrics commonly
used to compare microbial communities.

\textbf{The impact of removing rare ASVs on the probability of falsely
detecting a difference between treatment groups.} Observing reduced
ability to detect differences between communities when rare ASVs were
removed from each sample, I next asked whether removing rare ASVs could
lead to falsely claiming that a treatment effect had a significant
effect on community diversity and structure. First, I sampled sequences
from the null distribution for each dataset and randomly assigned each
sample to one of two treatment groups and determined the richness and
diversity of ASVs and OTUs. Testing at an experiment-wise error rate of
0.05, I expected 5\% of the iterations for each dataset to yield a
significant test result. Indeed, there was no evidence that removing
rare ASVs resulted in an inflated experiment-wise error rate. The
average fraction of significant tests did not meaningfully vary from
0.05 across the minimum abundance threshold, dataset, metric of
describing sample alpha-diversity, or whether the abundance of ASVs or
OTUs were used (Figure 5A and S6A). Similarly, the average fraction of
significant tests did not meaningfully vary from 0.05 when using
analysis of molecular variance to compare communities using Bray-Curtis
distances (Figure 5A and S6A). Second, I again sampled sequences from
the null distribution, but assigned samples to one of two treatment
groups based on the number of sequences in each sample. The samples with
fewer than the median number of sequences for the dataset were assigned
to one group and those with more than the median were assigned to the
other. This exaggerated bias has been observed in comparisons of the
lung and oral microbiota because of the larger number of non-specific
amplicons that can be sequenced from lung samples relative to those in
the oral cavity leading to a significant difference in sequencing depth
between treatment groups {[}\textbf{REF}{]}. When rare sequences were
not removed, the fraction of significant tests did not differ from 5\%
for comparing the richness, their diversity, or Bray-Curtis distances
(Figure 5B and S6B). However, when rare taxa of any frequency were
removed, the probability of falsely detecing a difference as signifiant
increased with the definition of rarity (Figure 5B and S6B). Not
including the small marine dataset, the average fraction the average
fraction of falsely detecting a difference across datasets when only
singletons were removed was 92.45\%. If there is any relationship
between the number of sequences and the treatment group, the risk of
falsely rejecting the null hypothesis is inflated when researchers use
the strategy of removing rare sequences. The most conservative approach
is to not remove low abundance sequences.

\textbf{\emph{Conclusion.}} Removing rare sequences decreases the
diversity represented by 16S rRNA gene sequence data and increases the
variation between samples. Such impacts will hinder the statistical
power to differentiate between treatment groups. Instead of removing
rare sequences, researchers should focus on optimizing their sequence
generation to minimize the amount of PCR and sequencing errors. In
addition, samples should be rarefied to a common number of sequences
across samples without prior culling of rare sequences. The number of
artifacts is correlated to the number of sequences being considered.
With this in mind, rarefaction allows one to control for uneven sampling
effort and to control for the number of artifacts in the analysis.

Need to treat rare sequences with a grain of salt

\newpage

\subsection{Acknowledgements}\label{acknowledgements}

\newpage

\subsection{Materials and Methods}\label{materials-and-methods}

\begin{itemize}
\itemsep1pt\parskip0pt\parsep0pt
\item
  sequencing pipeline description
\end{itemize}

\newpage

\subsection{References}\label{references}

\newpage

\textbf{Table 1. Summary of studies used in the analysis.} For all
studies, the number of sequences used from each study was rarefied to
the smallest sample size. A graphical represenation of the distribution
of sample sizes for each study and the samples that were removed from
each study are provided in Figure S1.

\begin{tabular}{lrrrrr}
\toprule
\textbf{Study (Ref)} & \textbf{Samples} & \textbf{\makecell[c]{Total\\sequences}} & \textbf{\makecell[c]{Median\\sequences}} & \textbf{\makecell[c]{Range of\\sequences}} & \textbf{\makecell[c]{Fold-difference\\between largest\\and smallest sample}}\\
\midrule
Bioethanol (NA) & 95 & 3,972,943 & 16,015 & 3,688-356,136 & 96.6\\
Human (NA) & 490 & 20,909,768 & 32,505 & 10,523-430,415 & 40.9\\
Lake (NA) & 52 & 3,169,868 & 69,041 & 15,347-112,871 & 7.4\\
Marine (NA) & 7 & 1,391,396 & 193,464 & 133,516-254,060 & 1.9\\
Mice (NA) & 348 & 2,813,747 & 6,477 & 1,804-30,565 & 16.9\\
Peromyscus (NA) & 111 & 1,555,545 & 12,446 & 4,464-33,644 & 7.5\\
Rainforest (NA) & 69 & 946,295 & 11,561 & 4,932-37,767 & 7.7\\
Rice (NA) & 490 & 22,591,168 & 43,216 & 2,776-193,464 & 69.7\\
Seagrass (NA) & 286 & 4,130,454 & 13,567 & 1,803-45,191 & 25.1\\
Sediment (NA) & 58 & 1,154,174 & 17,584 & 7,685-68,321 & 8.9\\
Soil (NA) & 18 & 956,656 & 51,844 & 47,806-59,956 & 1.3\\
Stream (NA) & 201 & 21,162,574 & 90,159 & 9,175-390,964 & 42.6\\
\bottomrule
\end{tabular}

\newpage

\textbf{Figure 1.}

\newpage

\textbf{Figure 2.} v

1. \textbf{Kozich JJ}, \textbf{Westcott SL}, \textbf{Baxter NT},
\textbf{Highlander SK}, \textbf{Schloss PD}. 2013. Development of a
dual-index sequencing strategy and curation pipeline for analyzing
amplicon sequence data on the MiSeq Illumina sequencing platform.
Applied and environmental microbiology \textbf{79}:5112--5120.

2. \textbf{Schloss PD}, \textbf{Westcott SL}, \textbf{Ryabin T},
\textbf{Hall JR}, \textbf{Hartmann M}, \textbf{Hollister EB},
\textbf{Lesniewski RA}, \textbf{Oakley BB}, \textbf{Parks DH},
\textbf{Robinson CJ}, \textbf{Sahl JW}, \textbf{Stres B},
\textbf{Thallinger GG}, \textbf{Horn DJV}, \textbf{Weber CF}. 2009.
Introducing mothur: Open-source, platform-independent,
community-supported software for describing and comparing microbial
communities. Applied and Environmental Microbiology
\textbf{75}:7537--7541.
doi:\href{http://dx.doi.org/10.1128/aem.01541-09}{10.1128/aem.01541-09}.

\end{document}
